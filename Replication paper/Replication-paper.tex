% AER-Article.tex for AEA last revised 22 June 2011
\documentclass[AER]{AEA}

% The mathtime package uses a Times font instead of Computer Modern.
% Uncomment the line below if you wish to use the mathtime package:
%\usepackage[cmbold]{mathtime}
% Note that miktex, by default, configures the mathtime package to use commercial fonts
% which you may not have. If you would like to use mathtime but you are seeing error
% messages about missing fonts (mtex.pfb, mtsy.pfb, or rmtmi.pfb) then please see
% the technical support document at http://www.aeaweb.org/templates/technical_support.pdf
% for instructions on fixing this problem.

% Note: you may use either harvard or natbib (but not both) to provide a wider
% variety of citation commands than latex supports natively. See below.

% Uncomment the next line to use the natbib package with bibtex
\usepackage{natbib}

% Uncomment the next line to use the harvard package with bibtex
%\usepackage[abbr]{harvard}

% This command determines the leading (vertical space between lines) in draft mode
% with 1.5 corresponding to "double" spacing.
\draftSpacing{1.5}

% For Pandoc highlighting needs

% Pandoc citation processing


\usepackage{hyperref}

\begin{document}

\title{The Power of Rumor Mill: How Rumors Shape Political Decision
Making}
\shortTitle{Replication of Berinsky (2017)}
% \author{Author1 and Author2\thanks{Surname1: affiliation1, address1, email1.
% Surname2: affiliation2, address2, email2. Acknowledgements}}


\author{
  Dan Xu\thanks{
  Xu: University of
Toronto, \href{mailto:zedan.xu@mail.utoronto.ca}{zedan.xu@mail.utoronto.ca}.
  Acknowledgements
}
}

\date{\today}
\pubMonth{10}
\pubYear{2021}
\pubVolume{1}
\pubIssue{1}
\JEL{A10, A11}
\Keywords{Rumors, Political decision making}

\begin{abstract}
This project is a replication of Berinsky's study of political rumors
surrounding the 2010 U.S. health care reform act. I re-examined the
structure of the experimental design and re-analyzed the results
generated by the survey experiment.
\end{abstract}


\maketitle

\section{Experimental design}

Berinsky conducted the survey experimental using a two wave online
survey. From 17 - 19 May 2010, Berinsky performed a between subjects
design experiment, in which the outcome is only measured post -
treatment, online with a national sample of 1701 American adults. The
second wave was administered to only 699 of the initial respondents from
25 - 29 May 2010. The experiment was conducted by Survey Sample
International (SSI) - a U.S. based digital research business that offers
survey sampling and related services for market survey research. The
survey was constructed according to the the US adult population on
education, gender, age, geography, and income.

\subsection{Between vs. Within Subject Design}

Widely used in Psychological studies, between-subject experimental
designs are an experimental approach in which each subject is tested
under one condition and only exposed to a single treatment. In
Berinsky's experimental design, the survey respondents were randomly
assigned to four different treatment groups and one control group and
presented with the stories regarding the controversial 2010 ACA. The
following table details the conditions which the survey respondents
received,

\begin{table}[h!]
\caption{The experimental treatments and control of the study}

\begin{tabular}{lll}
& Condition & Description \\
1 & Rumor & Rumor \\
2 & Rumor and Correction & Rumor and correction \\
3 & Rumor and Republican correction & Rumor and correction, Quote from Republican senator \\
4 & Rumor and Democratic correction & Rumor and correction, quote from emocratic representative \\
5 & Control & No experimeital conditions
\end{tabular}
\begin{tablenotes}
Table provides a brief description of the experimental design. See text for details.
\end{tablenotes}
\begin{tablenotes}[Source]
Table drawn on the original Berinsky(2017) article.
\end{tablenotes}
\end{table}

I would like to further clarify the differences between between-subject
(between - group) experimental designs and within-subjects (or
repeated-measures) experimental designs. In a between-subjects design,
or a between-groups design, each subject is only exposed to one
condition. In a within-subjects design, however, each participant
experiences all conditions.The same participants are tested repeatedly
in order to assess differences between conditions. Therefore, if
Berinsky were to conduct a within - subject experiment with the sample,
the survey respondents would not be assigned to five groups (four
treatment and one control). Instaed, they would be exposed to all the
same treatments to examine their reactions to the rumors.

Sample figure:

\begin{figure}
Figure here.

\caption{Caption for figure below.}
\begin{figurenotes}
Figure notes without optional leadin.
\end{figurenotes}
\begin{figurenotes}[Source]
Figure notes with optional leadin (Source, in this case).
\end{figurenotes}
\end{figure}

Sample table:

\begin{table}
\caption{Caption for table above.}

\begin{tabular}{lll}
& Heading 1 & Heading 2 \\
Row 1 & 1 & Rumor in the form of quotes endorsed by opponents \\
Row 2 & 3 & 4%
\end{tabular}
\begin{tablenotes}
Table notes environment without optional leadin.
\end{tablenotes}
\begin{tablenotes}[Source]
Table notes environment with optional leadin (Source, in this case).
\end{tablenotes}
\end{table}

References here (manual or bibTeX). If you are using bibTeX, add your
bib file name in place of BibFile in the bibliography command. \% Remove
or comment out the next two lines if you are not using bibtex.

\bibliographystyle{aea}
\bibliography{references}

\% The appendix command is issued once, prior to all appendices, if any.
\appendix

\section{Mathematical Appendix}

\end{document}
