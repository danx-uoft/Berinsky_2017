% AER-Article.tex for AEA last revised 22 June 2011
\documentclass[AER]{AEA}

% The mathtime package uses a Times font instead of Computer Modern.
% Uncomment the line below if you wish to use the mathtime package:
%\usepackage[cmbold]{mathtime}
% Note that miktex, by default, configures the mathtime package to use commercial fonts
% which you may not have. If you would like to use mathtime but you are seeing error
% messages about missing fonts (mtex.pfb, mtsy.pfb, or rmtmi.pfb) then please see
% the technical support document at http://www.aeaweb.org/templates/technical_support.pdf
% for instructions on fixing this problem.

% Note: you may use either harvard or natbib (but not both) to provide a wider
% variety of citation commands than latex supports natively. See below.

% Uncomment the next line to use the natbib package with bibtex
\usepackage{natbib}

% Uncomment the next line to use the harvard package with bibtex
%\usepackage[abbr]{harvard}

% This command determines the leading (vertical space between lines) in draft mode
% with 1.5 corresponding to "double" spacing.
\draftSpacing{1.5}

% For Pandoc highlighting needs

% Pandoc citation processing

\usepackage{booktabs}
\usepackage{longtable}
\usepackage{array}
\usepackage{multirow}
\usepackage{wrapfig}
\usepackage{float}
\usepackage{colortbl}
\usepackage{pdflscape}
\usepackage{tabu}
\usepackage{threeparttable}
\usepackage{threeparttablex}
\usepackage[normalem]{ulem}
\usepackage{makecell}
\usepackage{xcolor}

\usepackage{hyperref}

\begin{document}

\title{The Power of Rumor Mill: A replication Study of Berinsky's (2017)
Research on Political Rumors concerning the 2010 ACA}
\shortTitle{Replication of Berinsky (2017)}
% \author{Author1 and Author2\thanks{Surname1: affiliation1, address1, email1.
% Surname2: affiliation2, address2, email2. Acknowledgements}}


\author{
  Dan Xu\thanks{
  Xu: University of
Toronto, \href{mailto:zedan.xu@mail.utoronto.ca}{zedan.xu@mail.utoronto.ca}.
  Acknowledgements
}
}

\date{\today}
\pubMonth{10}
\pubYear{2021}
\pubVolume{1}
\pubIssue{1}
\JEL{A10, A11}
\Keywords{Rumors, Political decision making}

\begin{abstract}
This project is a replication of Berinsky's study of political rumors
surrounding the 2010 U.S. health care reform act. I re-examined the
structure of the experimental design and re-analyzed the results
generated by the survey experiment.
\end{abstract}


\maketitle

\hypertarget{experimental-design}{%
\section{Experimental design}\label{experimental-design}}

Berinsky conducted the survey experimental using a two wave online
survey. From 17 - 19 May 2010, Berinsky performed a between subjects
design experiment, in which the outcome is only measured post -
treatment, online with a national sample of 1701 American adults. The
second wave was administered to only 699 of the initial respondents from
25 - 29 May 2010. The experiment was conducted by Survey Sample
International (SSI) - a U.S. based digital research business that offers
survey sampling and related services for market survey research. The
survey was constructed according to the the US adult population on
education, gender, age, geography, and income.

\hypertarget{between-and-within-subject-design}{%
\subsection{Between and Within Subject
Design}\label{between-and-within-subject-design}}

Widely used in Psychological studies, between-subject experimental
designs are an experimental approach in which each subject is tested
under one condition and only exposed to a single treatment. In
Berinsky's experimental design, the survey respondents were randomly
assigned to four different treatment groups and one control group and
presented with the stories regarding the controversial 2010 ACA. The
following details the conditions which the survey respondents received,

1.Rumor only. In this group, respondents were presented with a rumor
surrounding the 2010 ACA.

2.Rumor and Correction. In this treatment group, respondents were
presented with a rumor and a correction debunking the rumor.

3.Rumor and Republican correction. In this treatment group, respondents
were presented with a rumor and correction, which was a quote from
Republican senator.

\begin{enumerate}
\def\labelenumi{\arabic{enumi}.}
\setcounter{enumi}{3}
\item
  Rumor and Democratic correction. In this condition, respondents were
  presented with a rumor and a correction, which was a uote from a
  Democratic representative.
\item
  Control. In the control group, no experimeital conditions were applied
  to the respondents.
\end{enumerate}

I would like to further clarify the differences between between-subject
(between - group) experimental designs and within-subjects (or
repeated-measures) experimental designs. In a between-subjects design,
or a between-groups design, each subject is only exposed to one
condition. In a within-subjects design, however, each participant
experiences all conditions.The same participants are tested repeatedly
in order to assess differences between conditions. Therefore, if
Berinsky were to conduct a within - subject experiment with the sample,
the survey respondents would not be assigned to five groups (four
treatment and one control). Instaed, they would be exposed to all the
same treatments to examine their reactions to the rumors.

\subsection{Structure of the experimental design  - Wave 1}

Even though there are 1701 respondents who were surveyed, only 1593
complete cases were documented. 99 respondents have not completed both
of the screeners. 1043 of the survey respondents passed Screener1, while
559 of them failed. Likewise, 1098 of the respondents passed the second
screener, while 504 failed. Approximately two thirds of the respondents
passed each of the screened questions, which confirms the author's
results. Furthermore, 879 of the respondents passed both of the screened
questions. Therefore, 54.9\% of the respondents whose responses were
recorded (excluding the incomplete cases that contain missing values)
passed both questions, which corresponds to the author's results. As he
states in the paper, 55\% of the sample passed both questions.

The following are the variables for the first-wave survey:

``ACA'' refers to ``Support for Health Care Reform.'' The question asks
``Overall, given what you know about them, would you say you support or
oppose the changes to the health care system that have been enacted by
Congress and the Obama administration?'' The answer choices are support
(1) and oppose(2) .

death refers to ``Death Panel Rumor.'' The qustion asks ``Do you think
the changes to the health care system have that have been enacted by
Congress and the Obama administration create ``death panels'' which have
the authority to determine whether or not a gravely ill or injured
person should receive health care based on their ``level of productivity
in society?'' The answer choices are "yes (1), no (2), not sure(3).

euth refers to ``Euthanasia Rumor.'' The question asks ``Do you think
the changes to the health care system have that have been enacted by
Congress and the Obama administration require elderly patients to meet
with government officials to discuss ``end of life'' options including
euthanasia?" The answer choices are "yes (1), no (2), not sure(3).

Screener 1 is the first of the two screened question the author used to
to measure respondent attention on the self-administered survey. For
this question, if the respondent passed the attention check, their
answer would be documented as ``1'', otherwise, ``0.''

Screener 2 is the second screened question for attention check. ``pass''
is coded as ``1'', and ``fail'' is coded as ``0.''

treat specifies the condition to which each group of respondents were
assigned. From 1 - 5, the groups are labelled as ``control'' ``rumor
only'' ``rumor+correct'' ``rumor+rep'',``rumor+dem.''

PID is a variable that documents one's party identification. From 0 - 6,
, it corresponds to ``str rep'' ``wk rep'' ``lean rep'' ``ind''
respectively, which translates to ``Strong Republican'',``Weak
Republican'',``Lean Republican'',``Independent'',``Lean
Democrat'',``Weak Democrat'',``Strong Democrat'' respectively.

The following analysis will be focused on the ``attentive sample,''
where the respondents passed both of the attention checks, according to
Berinsky.

\hypertarget{full-sample}{%
\section{Full sample}\label{full-sample}}

As stated earlier, the number of the complete cases for the dataset is
1593, instead of 1596. The researcher of the orginal study may need to
clarify why 1596 cases were used for analysis in the study.

\begin{table}

\caption{\label{tab:full sample}Treatment groups in the full sample}
\centering
\resizebox{\linewidth}{!}{
\begin{tabular}[t]{lrrrrr}
\toprule
  & Control & Rumor only & Rumor + Non - partisan Correction & Rumor + Republican Correction & Rumor + Democratic Correction\\
\midrule
Accept rumor & 57 & 64 & 58 & 60 & 63\\
Reject rumor & 170 & 140 & 177 & 183 & 167\\
Not sure & 110 & 109 & 79 & 75 & 81\\
\bottomrule
\end{tabular}}
\end{table}

Given the cross tabulation, I performed a Pearson chi-square test, which
is essentially a test to examine whether the results of a crosstab are
statistically significant. That is, if the two categorical variables
independent of one another. The statistics of the significance test are
as follows: N = 1593, \(\chi^2\)(8) = 18.819, Pr = 0.015, which are s
consistent with Berinsky's finding.

\begin{center}\includegraphics{Replication-paper_files/figure-latex/unnamed-chunk-6-1} \end{center}

The graph above also gives us an intuitive impression of the potential
relationship between the categorical variables.

\hypertarget{effects-of-treatments-on-euthanasia-rumor-belief}{%
\subsection{Effects of treatments on Euthanasia rumor
belief}\label{effects-of-treatments-on-euthanasia-rumor-belief}}

\begin{table}

\caption{\label{tab:t test}Effects of rumor on treatments}
\centering
\resizebox{\linewidth}{!}{
\begin{tabular}[t]{llrrr}
\toprule
treat & variable & n & mean & sd\\
\midrule
control & euth & 337 & 2.157 & 0.687\\
rumor only & euth & 313 & 2.144 & 0.731\\
rumor+correct & euth & 314 & 2.067 & 0.658\\
rumor+rep & euth & 318 & 2.047 & 0.651\\
rumor+dem & euth & 311 & 2.058 & 0.679\\
\bottomrule
\end{tabular}}
\end{table}

To investigate the effects of treatments on Euthanasis rumor belief, I
first presented a statistics summary table for each treatment group. As
seen in the table, there seems to be relatively little variation across
the five treatment groups. The means and standard deviations of the
conditions are similar, suggesting that the effects are minimal across
treatment groups.

\begin{center}\includegraphics{Replication-paper_files/figure-latex/unnamed-chunk-7-1} \end{center}

One of the most common methods for comparing means is t-test, which is
also the method Berinsky used to compared treatment group means. As the
graph shows, there was not a significant difference between the control
and treatment groups.

\begin{table}

\caption{\label{tab:unnamed-chunk-8}Significance test results between conditions}
\centering
\resizebox{\linewidth}{!}{
\begin{tabular}[t]{lllrrrrrrl}
\toprule
.y. & group1 & group2 & n1 & n2 & statistic & df & p & p.adj & p.adj.signif\\
\midrule
euth & control & rumor only & 337 & 313 & 0.2422141 & 636.4013 & 0.809 & 1.000 & ns\\
euth & control & rumor+correct & 337 & 314 & 1.7140886 & 648.5025 & 0.087 & 0.641 & ns\\
euth & control & rumor+rep & 337 & 318 & 2.1058857 & 652.9904 & 0.036 & 0.356 & ns\\
euth & control & rumor+dem & 337 & 311 & 1.8507636 & 642.9751 & 0.065 & 0.582 & ns\\
euth & rumor only & rumor+correct & 313 & 314 & 1.3843712 & 617.9056 & 0.167 & 0.835 & ns\\
\addlinespace
euth & rumor only & rumor+rep & 313 & 318 & 1.7527597 & 618.4225 & 0.080 & 0.641 & ns\\
euth & rumor only & rumor+dem & 313 & 311 & 1.5212066 & 619.2563 & 0.129 & 0.774 & ns\\
euth & rumor+correct & rumor+rep & 314 & 318 & 0.3784739 & 629.6417 & 0.705 & 1.000 & ns\\
euth & rumor+correct & rumor+dem & 314 & 311 & 0.1682396 & 621.9619 & 0.866 & 1.000 & ns\\
euth & rumor+rep & rumor+dem & 318 & 311 & -0.2018223 & 624.3881 & 0.840 & 1.000 & ns\\
\bottomrule
\end{tabular}}
\end{table}

Berinsky conducted two sets of Chi-squared Tests of Independence, also
refereed to as a \(\chi^2\) tests, to examine the significance
relationships across conditions. The first \(\chi^2\) test was to
examine the overall significance in order to examine if any significant
differences among any of the conditions exist. Likewise, I further
conducted a significance test to investigate the relationships between
groups. Given that there are five conditions, therefore, ten pairs of
comparisons were calculated. By comparing one condition to antoher, as
the table shows, the p values are greater than 0.05, and therefore, I
conclude that there are no relationships between each groups. In other
words, a heterogenous treamment effect between the groups does not exist
accorrding to the statistical results. However, Berinsky identified one
significant relationship amongst the ten pairs, which is that the
``rumor only'' condition is statistically significantly different from
the ``rumor + Republican correction''. My results suggest otherwise,
partisanship is not a factor in rumor acceptance in this particular
case. The Republican corretion is not a powerful treatment for rumor
belief.

\hypertarget{attentive-sample}{%
\section{Attentive sample}\label{attentive-sample}}

To ensure that the respondents paid close attention to the survey
experiment, Berinsky applied two attention checks in this study. The
attention check questions are designed to make sure that survey takers'
attention is at a high level throughout the entire survey. For those who
have succefully and correctly answered both of the questions, Berinsky
further categorized them into a subsample, which is called ``ateentive
sample'' accordingly.

Likewise, significant tests were performed on the subsample, which only
includes complete cases where both of the attention checks are marked as
``pass''.In total, there were 879 respondents who passed both of the
attention checks.

\begin{table}

\caption{\label{tab:attentive sample}Attentive sample}
\centering
\resizebox{\linewidth}{!}{
\begin{tabular}[t]{lrrrrr}
\toprule
  & Control & Rumor only & Rumor + Non - partisan Correction & Rumor + Republican Correction & Rumor + Democratic Correction\\
\midrule
Support & 94 & 65 & 81 & 87 & 65\\
Oppose & 91 & 90 & 96 & 94 & 113\\
\bottomrule
\end{tabular}}
\end{table}

By performing a Chi-square test, the following statistics were reported:
N = 879. X-squared = 23.946, df = 8, p-value = 0.00234, which are
consistent with Berinsky's finding.

\begin{center}\includegraphics{Replication-paper_files/figure-latex/unnamed-chunk-10-1} \end{center}

\begin{table}

\caption{\label{tab:unnamed-chunk-10}Significance tests of attentive sample}
\centering
\resizebox{\linewidth}{!}{
\begin{tabular}[t]{lllrrrrrrl}
\toprule
.y. & group1 & group2 & n1 & n2 & statistic & df & p & p.adj & p.adj.signif\\
\midrule
euth & control & rumor only & 185 & 155 & -1.6370006 & 329.0199 & 0.103 & 0.721 & ns\\
euth & control & rumor+correct & 185 & 177 & -0.9593972 & 359.3963 & 0.338 & 1.000 & ns\\
euth & control & rumor+rep & 185 & 181 & -0.5238201 & 363.8340 & 0.601 & 1.000 & ns\\
euth & control & rumor+dem & 185 & 178 & -2.7672907 & 360.9995 & 0.006 & 0.059 & ns\\
euth & rumor only & rumor+correct & 155 & 177 & 0.6997569 & 324.9972 & 0.485 & 1.000 & ns\\
\addlinespace
euth & rumor only & rumor+rep & 155 & 181 & 1.1253546 & 327.2372 & 0.261 & 1.000 & ns\\
euth & rumor only & rumor+dem & 155 & 178 & -1.0078028 & 322.3540 & 0.314 & 1.000 & ns\\
euth & rumor+correct & rumor+rep & 177 & 181 & 0.4355653 & 355.8630 & 0.663 & 1.000 & ns\\
euth & rumor+correct & rumor+dem & 177 & 178 & -1.7728210 & 352.4419 & 0.077 & 0.617 & ns\\
euth & rumor+rep & rumor+dem & 181 & 178 & -2.2241205 & 356.8546 & 0.027 & 0.241 & ns\\
\bottomrule
\end{tabular}}
\end{table}

For the sub-sample, the attentive sample according to Berinsky, I
performed significance tests amongst the ten pairs of treatment groups,
in order to examine the relationship between the treatment groups and
the effects of treatments on belief of the Euthanasia rumor. As the t -
test and significance test results show, there is no statitically
significant relationship between the two variables.

\hypertarget{effect-of-treatments-on-health-care-policy-opinion-attentive-sample}{%
\subsection{Effect of treatments on health care policy opinion
(attentive
sample)}\label{effect-of-treatments-on-health-care-policy-opinion-attentive-sample}}

\begin{table}

\caption{\label{tab:unnamed-chunk-11}Health care policy opinion (attentive sample)}
\centering
\resizebox{\linewidth}{!}{
\begin{tabular}[t]{lrrrrr}
\toprule
  & Control & Rumor only & Rumor + Non - partisan Correction & Rumor + Republican Correction & Rumor + Democratic Correction\\
\midrule
Support & 94 & 65 & 81 & 87 & 65\\
Oppose & 91 & 90 & 96 & 94 & 113\\
\bottomrule
\end{tabular}}
\end{table}

The above table shows the health care policy opinion by each treatment
group for the attentive sample.

The statistics of the Chi-square test are reported as follows, N = 879,
\(\chi ^2\) = 9.0038, p-value = 0.061, which is greater than 0.05. Such
statisticals are consistent with Berinsky's finding.

\begin{table}

\caption{\label{tab:t test ACA}Statistics of support for Health care policy reform (attentive sample)}
\centering
\resizebox{\linewidth}{!}{
\begin{tabular}[t]{llrrr}
\toprule
treat & variable & n & mean & sd\\
\midrule
control & ACA & 185 & 1.492 & 0.501\\
rumor only & ACA & 155 & 1.581 & 0.495\\
rumor+correct & ACA & 177 & 1.542 & 0.500\\
rumor+rep & ACA & 181 & 1.519 & 0.501\\
rumor+dem & ACA & 178 & 1.635 & 0.483\\
\bottomrule
\end{tabular}}
\end{table}

\includegraphics{Replication-paper_files/figure-latex/t test ACA-1.pdf}

\begin{table}

\caption{\label{tab:t test ACA}Health care policy opinion (attentive sample)}
\centering
\resizebox{\linewidth}{!}{
\begin{tabular}[t]{lllrrrrrrl}
\toprule
.y. & group1 & group2 & n1 & n2 & statistic & df & p & p.adj & p.adj.signif\\
\midrule
ACA & control & rumor only & 185 & 155 & -1.6370006 & 329.0199 & 0.103 & 0.721 & ns\\
ACA & control & rumor+correct & 185 & 177 & -0.9593972 & 359.3963 & 0.338 & 1.000 & ns\\
ACA & control & rumor+rep & 185 & 181 & -0.5238201 & 363.8340 & 0.601 & 1.000 & ns\\
ACA & control & rumor+dem & 185 & 178 & -2.7672907 & 360.9995 & 0.006 & 0.059 & ns\\
ACA & rumor only & rumor+correct & 155 & 177 & 0.6997569 & 324.9972 & 0.485 & 1.000 & ns\\
\addlinespace
ACA & rumor only & rumor+rep & 155 & 181 & 1.1253546 & 327.2372 & 0.261 & 1.000 & ns\\
ACA & rumor only & rumor+dem & 155 & 178 & -1.0078028 & 322.3540 & 0.314 & 1.000 & ns\\
ACA & rumor+correct & rumor+rep & 177 & 181 & 0.4355653 & 355.8630 & 0.663 & 1.000 & ns\\
ACA & rumor+correct & rumor+dem & 177 & 178 & -1.7728210 & 352.4419 & 0.077 & 0.617 & ns\\
ACA & rumor+rep & rumor+dem & 181 & 178 & -2.2241205 & 356.8546 & 0.027 & 0.241 & ns\\
\bottomrule
\end{tabular}}
\end{table}

Again, none of the relationships are statitically significant. Overall
the results are consistent with Berinsky's findings. However, he did
identified one statistically significant relationship between ``rumor
only'' and ``rumor + Republican correction.'' He also considered the
pairs with p values that are slightly greater than 0.05 marginally
statitically significant. My statitcal results suggest that none of the
relationships are considered statitically significant in Study 1.

\hypertarget{study-2}{%
\section{Study 2}\label{study-2}}

A few months after the first study, in which Berinsky surveyed 1701
American adults for their opinions on a political rumor surrounding the
ACA 2010, Berinsky conducted a second experiment on October/Noverber
2010. However, the experiment was administered by YouGov instead of
Survey Sampling International this time. The subject of the second
experiment was still rumors concerning the ACA 2010. According to
Berinsky, the stories presented to the respondents were modeled on the
stories from Study 1. However, the conditions were modified. The control
condition was identical to the ``rumor'' condition from Study 1. The
second ``correction only'' was altered and did not mention the rumor
concerning the death penal, but only with the description of the
provisions in the 2010 ACA. The third condition that was identical to
the ``rumor and Republican correction'' condiction from Study 1 and
presented both the rumor from the first condition and the correction
from the second condition. For Study 2, Berinsky did not apply attention
checks. In addition, no control group was included in this follow - up
study.

In this study, the respondents were presented with two types of recall
questions, where they were asked to identify the person who said the
quote. The sample was split two two groups for comparison. Half of the
survey takers were assigned to an ``irrelevant'' recall" condition, in
which the question asks what offce was held by Bestsy McCaughey, while
the other half was placed in a ``long recall'' condition, where they
received two questions with identical answer choices. The first recall
question asks ``You have every right to fear\ldots{[}You{]} should not
have a government-run plan to decide when to pull the plug on Grandma.''
The second recall questions asks ``The health care reform bill requires
``people in Medicare have a required counseling session that will tell
them how to end their life sooner.''

This dataset contains all the 1000 cases Berinsky collected for the
second study, which is referred to as CCES. The following are the
variables,

``randtreat'' refers to the story condition, including the three
conditions, namely `Rumor only',`Correction only',`Rumor +
Correction'.Please note that no control group is included in this study.

``randtreat2'' refers to the first recall condition, including `Short
recall' and `Long recall' questions

"``ACA'' refers to whether the respondent supports or opposes the act.

``euth\_w1'' refers to whether the Euthanasia rumor was belieavable in
Wave 1.

``euth\_w2'' refers to whether the Euthanasia rumor was belieavable in
Wave 2.

Please note that Berinsky conducted two waves of the study, in October
and November, respectively, and the belief question was asked in both
waves. Therefore, ``euth\_w1'' am ``euth\_w2'' ask the same question.
However, support for the health care reform bill (``ACA'') was only
asked in Wave 1.

\begin{table}

\caption{\label{tab:unnamed-chunk-13}Study 2 (Wave 1)}
\centering
\resizebox{\linewidth}{!}{
\begin{tabular}[t]{lrrr}
\toprule
  & Rumor only & Correction only & Rumor + Correction\\
\midrule
Accept rumor & 102 & 71 & 67\\
Reject rumor & 168 & 167 & 206\\
Not sure & 80 & 70 & 69\\
\bottomrule
\end{tabular}}
\end{table}

In general, we can observe from the table that respondents tended to
reject the rumor. The statistics of the Chi - square test are consistent
with Berinsky's finding: N = 1000, \(\chi^2\) (4) = 12.237, p-value =
0.01567, which is smaller than 0.05, indicating that the relationship is
significant.

\begin{table}

\caption{\label{tab:unnamed-chunk-14}Recall effects in Study 2 (Wave 2)}
\centering
\resizebox{\linewidth}{!}{
\begin{tabular}[t]{lrr}
\toprule
  & Short recall & Long recall\\
\midrule
Strongly support & 105 & 94\\
Somewhat support & 124 & 111\\
Somewhat oppose & 80 & 72\\
Strongly oppose & 192 & 221\\
\bottomrule
\end{tabular}}
\end{table}

The Chi square test statistics are X-squared = 3.7756, df = 3, p-value =
0.2867, suggesting that the relationship is not statitically
significant. That is, there is no difference between the effects of
irellevant or long recall questions. Respodents exposed to either
conditions showed no difference in accepting political rumors.

To gauge the number of respondents who completed both waves, I used the
variables euth\_w1 and euth\_w2 in the dataset provided by Berinsky.
Given that all the 1000 respondents have provided a valid response for
the question concerning euth\_w1, I can conclude that the number of
respondents who completed the first wave is 1000, and the the number of
the respondents who completed both waves depends on the second wave. 165
Respondents have not provided a answer to the question concerning
euth\_w2, therefore, the number of respondents who completed both waves
should be 835 instead of 837.

\begin{table}

\caption{\label{tab:unnamed-chunk-15}Recall effects in Study 2 (Wave 1)}
\centering
\resizebox{\linewidth}{!}{
\begin{tabular}[t]{lrrr}
\toprule
  & Rumor only & Correction only & Rumor + Correction\\
\midrule
Accept rumor & 79 & 61 & 52\\
Reject rumor & 137 & 144 & 181\\
Not sure & 73 & 56 & 52\\
\bottomrule
\end{tabular}}
\end{table}

The statistics are: X-squared = 15.172, df = 4, p-value = 0.004357,
which is less than 0.05, suggesting the overall significance between the
three treatment conditions. This finding is consistent with Berinsky's.

\begin{table}

\caption{\label{tab:unnamed-chunk-16}Recall effects in Study 2 (Wave 2)}
\centering
\resizebox{\linewidth}{!}{
\begin{tabular}[t]{lrrr}
\toprule
  & Rumor only & Correction only & Rumor + Correction\\
\midrule
Accept rumor & 76 & 68 & 66\\
Reject rumor & 135 & 138 & 156\\
Not sure & 78 & 55 & 63\\
\bottomrule
\end{tabular}}
\end{table}

The statistics are X-squared = 5.0398, df = 4, p-value = 0.2832. Which
again confirms Berinsky's finding.

\hypertarget{reheasal-effect}{%
\subsection{reheasal effect}\label{reheasal-effect}}

\begin{table}

\caption{\label{tab:unnamed-chunk-18}Recall effects on rumor only conditiion in Study 2 (Wave 2)}
\centering
\resizebox{\linewidth}{!}{
\begin{tabular}[t]{lrr}
\toprule
  & Short recall & Long recall\\
\midrule
Accept rumor & 35 & 41\\
Reject rumor & 74 & 61\\
Not sure & 32 & 46\\
\bottomrule
\end{tabular}}
\end{table}

For ``rumor only'' condition, the statistics are N= 289, X-squared =
4.0712, df = 2, p-value = 0.1306.

\begin{table}

\caption{\label{tab:unnamed-chunk-19}Recall effects on rumor + correction conditiion in Study 2 (Wave 2)}
\centering
\resizebox{\linewidth}{!}{
\begin{tabular}[t]{lrr}
\toprule
  & Short recall & Long recall\\
\midrule
Accept rumor & 31 & 35\\
Reject rumor & 84 & 72\\
Not sure & 27 & 36\\
\bottomrule
\end{tabular}}
\end{table}

The statistics for the ``rumor + correction'' condition are N = 285,
X-squared = 2.4477, df = 2, p-value = 0.2941

Again it seems that there is no statistically significant relationship
between the ``recall'' condition and rumor acceptance. My results are
consistent with Berinsky's, merely increasing the fuency of the rumor
increases its effectiveness.

\hypertarget{discussion-and-conclusion}{%
\section{Discussion and conclusion}\label{discussion-and-conclusion}}

Overall my results are consistent with Berinsky's findings, suggesting
that the studied Berinsky conducted are replicable.

For Study 1, Berinsky concludes that corrections from Republicans
debunking heath care rumors are the most effecitve way to conter
misinformation. However, my replicated results show that none of the
relationships amongst the control and treatment groups are significant.
In other words, my results do not lend support to Berinsky's conclusion.
We did not identify other statitically significant relationships betwen
the treatment groups. However, the results of the t - tests and
significance tests I conducted lend strong support to the conclusioin
that there is no statistically significant relationship among the
variables.

For Study 2, our results are consisitent. My results show that Study 2
replicates the findings from Study 1, no statitically significant
relationships existed among the treatment groups. The hypothesis that
fluency of the rumor may increase effectiveness of rumor spreading does
not gain sufficient support.

\bibliographystyle{aea}
\bibliography{references}

\end{document}
